\documentclass{article}

% Language setting
% Replace `english' with e.g. `spanish' to change the document language
\usepackage[english]{babel}

% Set page size and margins
% Replace `letterpaper' with `a4paper' for UK/EU standard size
\usepackage[letterpaper,top=2cm,bottom=2cm,left=3cm,right=3cm,marginparwidth=1.75cm]{geometry}

% Useful packages
\usepackage{amsmath}
\usepackage{graphicx}
\usepackage[colorlinks=true, allcolors=blue]{hyperref}

\title{Your Paper}
\author{You}

\begin{document}
\maketitle

\begin{abstract}
Your abstract.
\end{abstract}

\section{Introduction}
Aqui está a composição dos principais mésons conhecidos:
\newline
\newline
Mésons Pseudoscalars
\newline
\newline
- Pion ($\pi$):

    - $\pi^+$: $u\overline{d}$
    
    - $\pi^0$: $\frac{u\overline{u} - d\overline{d}}{\sqrt{2}}$
    
    - $\pi^-$: $d\overline{u}$
\newline
\newline
- Kaon ($K$):

    - $K^+$: $u\overline{s}$
    
    - $K^0$: $d\overline{s}$
    
    - $\overline{K^0}$: $s\overline{d}$
    
    - $K^-$: $s\overline{u}$
\newline
\newline
- Eta ($\eta$):

    - $\eta$: $\frac{u\overline{u} + d\overline{d} - 2s\overline{s}}{\sqrt{6}}$
    
    - $\eta'$: $\frac{u\overline{u} + d\overline{d} + s\overline{s}}{\sqrt{3}}$
\newline
\newline
Mésons Vetoriais
\newline
\newline
- Rho ($\rho$):

    - $\rho^+$: $u\overline{d}$
    
    - $\rho^0$: $\frac{u\overline{u} - d\overline{d}}{\sqrt{2}}$
    
    - $\rho^-$: $d\overline{u}$
\newline
\newline
- Omega ($\omega$):

    - $\omega$: $\frac{u\overline{u} + d\overline{d}}{\sqrt{2}}$
\newline
\newline
- Phi ($\phi$):

    - $\phi$: $s\overline{s}$
\newline
\newline
- J/Psi ($J/\psi$):

    - $J/\psi$: $c\overline{c}$
\newline
\newline
- Upsilon ($\Upsilon$):

    - $\Upsilon$: $b\overline{b}$
\newline
\newline
Outros Mésons
\newline
\newline
- D:

    - $D^+$: $c\overline{d}$
    
    - $D^0$: $c\overline{u}$
    
    - $\overline{D^0}$: $u\overline{c}$
    
    - $D^-$: $d\overline{c}$
\newline
\newline
- B:

    - $B^+$: $u\overline{b}$
    
    - $B^0$: $d\overline{b}$
    
    - $\overline{B^0}$: $b\overline{d}$
    
    - $B^-$: $b\overline{u}$
\newline
\newline
Essas composições refletem a combinação de quarks e antiquarks que formam cada méson. Se precisar de mais detalhes ou tiver outras perguntas, estou aqui para ajudar!

\bibliographystyle{alpha}
\bibliography{sample}

\end{document}